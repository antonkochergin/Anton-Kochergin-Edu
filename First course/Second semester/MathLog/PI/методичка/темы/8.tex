\section{Аксиоматическое построение логики высказываний (исчисление высказываний)}
\subsection{Аксиомы и правила вывода исчисления высказываний}
Множество теорем $Ax(\text{ИВ})$ описывается следующими тремя аксиомами:

\begin{enumerate}
    \item $(A_1)$: $\Phi \then (\Psi \then \Phi)$
    \item $(A_1)$: $(\Phi \then (\Psi \then \Xi)) \then ((\Phi \then \Psi) \then (\Phi \then \Xi))$
    \item $(A_2)$: $(\lnot \Phi \then \lnot \Psi) \then ((\lnot \Phi \then \Psi) \then \Phi)$
\end{enumerate}

\dftion Правило вывода исчисления высказываний --- \textbf{правило заключения} (modus ponens):

$$
MP: \begin{matrix}
    \Phi \then \Psi, \Phi \\
    \Psi
\end{matrix}
$$
\subsection{Доказуемость формул}
\dftion Формула $\Phi$ --- \textbf{теорема исчисления высказываний}, если найдётся конечная последовательность $\xses[\Phi]$ такая, что

\begin{enumerate}
    \item $\Phi_n = \Phi$
    \item Каждая формула $\Phi_i$ либо является аксиомой, либо получается из некоторых двух предыдущих формул $\Phi_j, \Phi_k$ по правилу вывода MP.
\end{enumerate}

Последовательность формул $\xses[\Phi]$ называется \textbf{выводом} или \textbf{доказательством} формулы $\Phi$.

\dftion Вывод формулы $\Phi$ сокращённо обозначают символом $\vdash$ и говорят, что $\Phi$ есть теорема. Множество всех таких теорем обозначается символом $Th(\text{ИВ})$ и называется \textbf{теорией исичления высказываний}.
\subsection{Тождественная истинность выводимых формул}
\textbf{Лемма}. Справедливы следующие утверждения:

\begin{enumerate}
    \item Аксиомы ИВ являются тавтологиями
    \item Результат применения правила вывода MP к любым тавтологиям $\Phi \then \Psi, \Phi$ даёт тавтологию $\Psi$
    \item Любая теорема ИВ является тавтологией, то есть выполняется $T_{\text{АВ}} \subset Th(\text{ИВ})$
\end{enumerate}

\subsection{Теорема Геделя о полноте исчисления высказываний}
Формула является выводимой в исчислении высказываний тогда и только тогда, когда она общезначима (истинна в любой интерпретации при любой подстановке).

\subsection{Непротивроечивость, полнота и разрешимость исчисления высказываний}
\textbf{Теорема о полноте}. Любая тавтология является теоремой ИВ, а любая теорема ИВ является тавтологией, следовательно, $T_{\text{АВ}} = Th(\text{ИВ})$.

\textbf{Теорема о непротивроечивости}. В исчислении высказываний невозможно доказать никакую формулу $\Phi$ вместе с её отрицанием $\lnot \Phi$.

\textbf{Теорема о разрешимости}. Существует алгоритм, который для любой формулы определяет, является ли эта формула теоремой ИВ.