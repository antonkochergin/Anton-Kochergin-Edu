\section{Автоматическое доказательство теорем}
\subsection{Нормальные формы формул алгебры предикатов}
В алгебре предикатов так же, как и в алгебре высказываний, существуют нормальные формы, которые аналогичны ДНФ и КНФ.

\dftion \textbf{Предварённая (пренексная) нормальная форма} (ПНФ) --- формула вида $(K_1 x_1)\dots(K_n x_n)\Psi$, где $\xses[K]$ --- кванторы, $\Psi$ --- бескванторная формула в КНФ. Последовательность кванторов называется \textit{кванторной приставкой}, а $\Psi$ --- \textit{конъюнктивным ядром} формулы $\Phi$.

\underline{Теорема 1}. Любая формула перечисления предикатов $\Phi$ логически равносильна формуле $\Phi'$ в ПНФ, причём $\Phi'$ называется \textbf{пренексной нормальной формулой} формулы $\Phi$.

\subsection{Скулемовские функции и приведение формул к скулемовской стандартной форме}
\dftion \textbf{Скулемовская стандартная форма} (ССФ) --- частный случай ПНФ, при котором в кванторной приставке содержатся только кванторы $\forall$. Приведение к ССФ осуществляется согласн стандартному алгоритму:

\begin{enumerate}
    \item Если левее $(\exists x)$ не стоит кванторов общности, $x$ в конъюнктивном ядре заменяется новым предметным символом $c$, а квантор убирается
    \item Если же там стоят кванторы $(\forall x_{s_1})\dots(\forall x_{s_m})$, все вхождения заменя.ься $x$ на новый $m$-арный функциональный символ $f(x_{s_1},\dots,x_{s_m})$ и убирается квантор $(\exists x)$
\end{enumerate}

\subsection{Сведение проблемы общезначимости формул к проблеме противоречивости систем дизъюнктов}
Общезначимость формулы $\Phi$ доказывается попыткой построения интерпретации, опровергающей эту формулу. Если по пути мы приходим к противоречию, мы доказываем, что такой интерпретации не существует, то есть $\Phi$ общезначима. В противном случае мы находим контрпример, доказывающий, что $\Phi$ не общезначима.
\subsection{Подстановки и унификация формул}
\dftion \textbf{Термы} --- выражения языка, которые индуктивно определяются следующим образом:

\begin{itemize}
    \item Все предметные переменные и предметные символы формулы являются термами
    \item Если $f$ --- $n$-арный функциональный символ формулы и $t_1, \dots, t_n$ --- термы, то $f(t_1,\dots,t_n)$ --- тоже терм
\end{itemize}

Определим $X_S, C_S, F_S$ как предметные переменые, предметные символы и функциональные символы, встречающиеся во множестве формул алгебры предикатов $S$. Пусть $A_S = X_S \cup C_S$. Если $C_S = \varnothing$, то к $A_S$ добавляется постоянный символ $a$. Определим на $A_S$ множество термов $T_S$, в которое входят функциональные символы из множества $F_S$ и каждая переменная из $X_S$.

\dftion \textbf{Подстановки} --- отображения $\theta: X_S \to T_S$. Обозначаются

$$\theta = \left(
\begin{matrix}
    x_1 & \dots & x_n \\
    t_1 & \dots & t_n
\end{matrix}
\right),$$

где $\forall x_i \in dom \, \theta \; \,  \theta(x_i) \neq x_i \then \theta \; t_i=\theta(x_i)$

Пусть $W=\{\Phi_1,\dots,\Phi_k\}$ --- множество атомарных формул, встречающихся в формулах из множества $S$. Подстановка $\theta$ называется \textit{унификатором множества формул} $W$, если $\theta(\Phi_1) = \dots = \theta(\Phi_k$).

Говорят, что множество атомарных формул $W$ \textit{унифицируемо}, если для него существует унификатор.
\subsection{Метод резолюций в логике предикатов}
\dftion \textbf{Резолютивный вывод} формулы $\Phi$ из множества дизъюнктов $S$ --- конечная последовательность дизъюнктов $\Phi_1, \dots, \Phi_k$, что

\begin{enumerate}
    \item $\Phi_k = \Phi$,
    \item Каждый дизъюнкт $\Phi_i$ или принадлежит множеству $S$, или является резольвентой некоторых дизъюнктов, предшествующих $\Phi_i$.
\end{enumerate}

Резолютивный вывод из множества дизъюнктов $S$ сохраняет выполнимость формул.
\subsection{Основная теорема метода резолюций в логике предикатов --- теорема полноты резолютивного вывода}
\textbf{Теорема}. Множество дизъюнктов $S$ противоречиво тогда и только тогда, когда существует резоллютивный вывод нуля из $S$.