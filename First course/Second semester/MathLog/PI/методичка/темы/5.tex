\section{Алгебра предикатов}
\subsection{Понятие предиката}
\dftion \textbf{Предикат} --- утверждение, содержащее \textit{предикативные} переменные $x_1,\dots,x_n$, которое превращается в высказывание при замене переменных конкретными объектами из области возможных значений $M$. Предикат с $n$ переменными называется $n$-арным или $n$-местными и обозначается $P(x_1,\dots,x_n)$.

Истинностная функция предиката $F_P : M^n \to \{0,1\}$ определяется множеством истинности $P^+ = \{(a_1, \dots, a_n) \in M^n : \lambda(P(a_1,\dots,a_n)) = 1\}$.
\subsection{Перенесение на предикаты логических операций}
\dftion \textbf{Отрицание} $n$-местного предиката --- $n$-местный предикат, который при подстановке значений превращается в высказывание, являющееся отрицанием высказывания исходного предиката.

\dftion \textbf{Конъюнкция} (\textbf{дизъюнкция}) $n$-местных предикатов $P(x_1,\dots,$ $x_n)$, $Q(x_1,\dots,x_n)$ --- $n$-местный предикат, который при подстановке значений превращается в высказывание $P\land Q(x_1,\dots,x_n)$ \Big($P\lor Q(x_1,\dots,x_n)$\Big).

\subsection{Множество истинности предикатов, полученных при помощи логических операций}
\begin{itemize}
    \item $(\lnot P)^+ = M \backslash P^+$
    \item $(P \land Q)^+ = P^+ \cap Q^+$
    \item $(P \lor Q)^+ = P^+ \cup Q^+$
    \item $(P \then Q)^+ = (\lnot P)^+ \cup Q^+$
    \item $(P \eq Q)^+ = (P \then Q)^+ \cap (Q \then P)^+$
\end{itemize}

\subsection{Кванторы общности и существования, их действие на предикат}
\dftion $\forall$ --- квантор общности, $\exists$ --- квантор существования.

$n-1$-местный предикат $(\forall x_1)P(x_1,\dots,x_n)$ зависит от переменных $x_2,\dots,x_n$ и при значениях $x_2=a_2,\dots,x_n=a_n$ истинен тогда и только тогда, когда при любых значениях $x_1 = a_1 \in M$ высказывания $P(a_1,a_2,\dots,a_n)$ истинны.

$n-1$-местный предикат $(\exists x_1)P(x_1,\dots,x_n)$ зависит от переменных $x_2,\dots,x_n$ и при значениях $x_2=a_2,\dots,x_n=a_n$ истинен тогда и только тогда, когда существует такое значение $x_1 = a_1 \in M$, при котором высказывание $P(a_1,a_2,\dots,a_n)$ истинно.

\subsection{Свободные и связанные переменные}
\dftion \textbf{Связные} переменные --- переменные, которые находятся в области действия одного из кванторов. Все прочие переменные называются \textbf{свободными}

\subsection{Формулы алгебры предикатов}
\dftion Алфавит алгебры предикатов $\mathscr{P}$ состоит из следующих символов:
\begin{enumerate}
    \item предикативные переменные $x_1, x_2, \dots$
    \item $n$-местные предикатные симвлоы $P, Q, \dots$
    \item символы логических операций $\lnot, \land, \lor, \then, \eq, \forall, \exists$
\end{enumerate}