\section{Аксиоматическое построение логики предикатов (исчисление предикатов)}
\subsection{Аксиомы и правила вывода исчисления предикатов}
Множество $Ax(\text{ИП})$ содержит в себе пять аксиом:

\begin{enumerate}
    \item $(A_1)$: $\Phi \then (\Psi \then \Phi)$
    \item $(A_2)$: $(\Phi \then (\Psi \then \Xi)) \then ((\Phi \then \Psi) \then (\Psi \then \Xi))$
    \item $(A_3)$: $(\lnot \Phi \then \lnot \Psi) \then ((\lnot \Phi \then \Psi) \then \Phi)$
    \item $(A_4)$: $(\forall x)\Phi(x) \then \Phi(y)$ для формул $\Phi(x)$, в которые $x$ не входит связно
    \item $(A_5)$: $(\forall x)(\Phi \then \Psi(x)) \then (\Phi \then (\forall x)\Psi(x))$ для формул $\Phi$, в которые $x$ не входит свободно
\end{enumerate}

Исчисление предикатов имеет два правила вывода --- \textbf{правило заключения} (modus ponens) и \textbf{правило обобщения}:

\begin{figure}[H]
    \centering
    \begin{tabular*}{0.5\textwidth}{@{\extracolsep{\fill}}ccc@{}}
        $MP: \begin{matrix}
            \Phi \then \Psi, \Phi \\
            \Psi
        \end{matrix}$ &
        и &
        $Gen: \begin{matrix}
            \Phi \\
            (\forall x)\Phi
        \end{matrix}$.
    \end{tabular*}
\end{figure}

\dftion Формула $\Phi$ называется \textbf{теоремой исчисления предикатов}, если найдётся такая последовательность $\xses[\Phi]$, в которой

\begin{enumerate}
    \item $\Phi_n = \Phi$
    \item Любая формула $\Phi_i$ либо является аксиомой, либо получается из некоторых предыдущих формул этой последовательности по одному из правил вывода $MP$ или $Gen$.
\end{enumerate}

При этом последовательность $\xses[\Phi]$ называется \textbf{выводом} или \textbf{доказательством} формулы $\Phi$.

\subsection{Тождественная истинность выводимых формул}
\textbf{Лемма 1}. Справедливы следующие утверждения:
\begin{enumerate}
    \item Всякая аксиома ИП является тавтологией
    \item Результат применения правил вывода MP и Gen к тавтологиям является тавтологией
    \item Любая теорема ИП является тавтологией ИП, т. е. имеет место $T_{\text{АП}} \subset Th(\text{ИП})$
\end{enumerate}

\subsection{Полнота и непротиворечивость исчисления предикатов}
\textbf{Теорема полноты ИП}.

Формула ИП является тавтологией, если она есть теорема ИП, и наоборот, то есть $T_{\text{АП}} = Th(\text{ИП})$

\textbf{Теорема о непротиворечивости ИП}.

В исчислении предикатов невозможно доказать никакую формулу $\Phi$ вместе с её отрицанием $\lnot \Phi$.

\textbf{Теорема о неразрешимости ИП}.

Не существует универсальной эффективной процедуры (алгоритма), которая для любой формулы определяет, является ли эта формула теоремой ИП.