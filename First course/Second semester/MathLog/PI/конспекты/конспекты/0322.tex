\chapter{22 марта. Интерпретация формул алгебры предикатов}


{\it Выполнимость формулы} $\Phi$ в интерпретации $M$ при оценке $\alpha$ обозначается $M \vDash_\alpha \Phi$ - читается "формула $\Phi$ истинна в интерпретации $M$ при оценке $\alpha$" и определяется следующим образом:
\begin{enumerate}
    \item $M \vDash_\alpha P(x_1, \dots, x_n)$ означает, что $P_M(\alpha(x_1), \dots, \alpha(x_n))$ - ист. выск.
    \item $M \vDash_\alpha \neg \Psi$ означает, что $M \cancel{\vDash_\alpha} \Psi$, т. е.
    \item $M \vDash_\alpha \Phi_1 \land \Phi_2$ тогда и только тогда, когда $M \vDash_\alpha \Phi_1$ и $M \vDash_\alpha \Phi_2$
    \item $M \vDash_\alpha = \Phi_1 \lor \Phi_2$ означает, что $M \vDash_{\alpha_1} \Phi_1$ или $M \vDash_\alpha \Phi_2$
    \item $M \vDash_\alpha = \Phi_1 \Rightarrow \Phi_2$ означает, что неверно $M \vDash_{\alpha_1} \Phi_1$ и $M \vDash_\alpha \lnot \Phi_2$
    \item $M \vDash_\alpha = \Phi_1 \Leftrightarrow \Phi_2$ означает, что $M \vDash_{\alpha_1} \Phi_1$. $M \vDash_\alpha \Phi_2$ одновременно верны или не верны
    \item $M \vDash_\alpha = (\forall x)\Psi$ означает, что $M \vDash_{\alpha_1} \Phi$, когда $M \vDash_{\alpha'} \Phi$ для любой оценки $\alpha'$, отличной от $\alpha$, возможно только на $x$
    \item $M \vDash_\alpha = (\exists x)\Psi$ означает, что $M \vDash_{\alpha_1} \Phi$ для любой оценки $\alpha'$, отличной от $\alpha$, возможно только на $x$
\end{enumerate}

\section{Классификация формул алгебры предикатов}


\dftion В интерпретации $M$ формула $\Phi$ называется:
\begin{itemize}
    \item {\it общезначимой} (тождественно истинной), если $M \vDash_\alpha \Phi$ при любых оценках $\alpha$
    \item {\it выполнимой}, если $M \vDash_\alpha \Phi$ для некоторой оценки $\alpha$
    \item {\it опревержимой}, если для некоторой оценки $\alpha$ неверно, что $M \vDash_\alpha \Phi$
    \item {\it тождественно ложной}, если для любой оценки $\alpha$ неверно, что $M \vDash_\alpha \Phi$
\end{itemize}


Формула $\Phi$ общезначима в интерпретации $M$ (с интерпретацией $P_M$ $n$-арных предикатных символов $P$), если она превращается в тождественно истинный на множестве $M$ предикат. Символическая запись $M \vDash \Phi$.

Формула $\Phi$ в интерпретации $M$ выполнима, опровержима или тождественно ложна, если она превращается соответственно в выполнимый, опровержимы или тождественно ложный намножестве $M$ предикат $P_M$.

$M \vDash \Phi$ означ., что $M \vDash_\alpha \Phi$ при любой оценке $\alpha$.

{\it Примеры.} \\
$M \vDash P(x) \Leftrightarrow Q(x)$ равносильно $P_M(\alpha(x)) \Leftrightarrow Q_M(\alpha(x))$, \\
$M \vDash P(x) \Rightarrow Q(x)$ равносильно $P_M(\alpha(x)) \Rightarrow Q_M(\alpha(x))$, \\ \\
$M \vDash P(x) \Leftrightarrow Q(x)$ равносильно $P_M^+ = Q_M^+$, \\
$M \vDash P(x) \Rightarrow Q(x)$ равносильно $P_M^+ \subset Q_M^+$, \\
$M \vDash (\forall x)P(x)$ равносильно $P_M^+ = M$, \\
$M \vDash (\forall exists)P(x)$ равносильно $P_M^+ \neq \varnothing$.


\dftion Формула $\Phi$ называется {\it тождественно истинной}, если она тождественно истинна в любой интерпретации $M$. Такая формула называется также {\it общезначимой формулой}, или {\it тавтологией алгебры предикатов} и обозначается $\vDash \Phi$. Множество всех тавтологий алгебры предикатов обозначим $\mathscr{T}_{\text{АП}}$

\dftion Формула $\Phi$ называется {\it тождественно ложной} или {\it противоречием}, если она тождественно ложна в любой интерпретации $M$.

По определению противоречивость формулы $\Phi$ равносильна условию $\vDash \neg \Phi$.

\dftion Формула $\Phi$ называется {\it выполнимой}, если она выполнима хотя бы в одной интерпретации $M$, которая называется {\it моделью} этой формулы.

{\it Пример 1.} Покажем, что $\vDash (\forall x)P(x) \Rightarrow (\exists x)P(x)$

Рассмотрим интерпретацию $M$ с предик. $P_M=P_M(x)$, для кат. $M \vDash (\forall x)P(x)$. Это означает, что $P_M^+ = M \neq \varnothing$, $\Psi$ - тавтология. Следовательно, $P_M^+ \neq \varnothing$ и выполняется $M \vDash (\exists x)P(x)$.

Значит, $M \vDash (\forall x)P(x) \Rightarrow (\exists x)P(x)$ для любой интерпретации $M$.


{\it Пример 2.} Покажем, что $\vDash (\exists x)P(x) \Rightarrow (\forall x)P(x)$ \\
Рассмотрим интерпретацию $M=\{a, b\}$ и предикат $P_M=(x=a)$, $P_M^+ = \{a\} \neq \varnothing, M$. Тогда на $M \vDash (\exists x)P(x)$, т. к. $P_M^+ \neq \varnothing$, но $M \cancel{\vDash}(\forall x)P(x)$, т. к. $P_M^+ \neq M$. В результате $M \cancel{\vDash} (\exists x)P(x) \Rightarrow (\forall x)P(x)$ и формула $(\exists x)P(x) \Rightarrow (\forall x)P(x)$ опровержима. \\
\\
Как мы видим на примерах, тождественная истинность и опровержимость доказывается по разному. Таким образом, формула $\Phi$:
\begin{itemize}
    \item общезначимая (или тождественно истинная, тавтология), если $M \vDash_\alpha \Phi$ в любой интерпретации $M$ при любых оценках $\alpha$; запись $\vDash\Phi$;
    \item выпонимая, если $M \vDash_\alpha \Phi$ в некоторой интерпретации $M$ для некоторой оценки $\alpha$
    \item опровержимая, если в некоторой интерпретации $M$ для некоторой оценки $\alpha$ неверно, что $M \vDash_\alpha \Phi$
    \item тождественно ложная, если в любой интерпретации $M$ для любой оценки $\alpha$ неверно, что $M \vDash_\alpha \Phi$
\end{itemize}

\underline{Замечание}. Если формула $\Phi$ является предложением, то она не содержит свободных вхождений переменных и, следовательно, не зависит от оценок $\alpha$ предметных переменных в области интерпретации $M$.

Значит, предложение $\Phi$ в интерпретации $M$ общезначимо в том и только в том случае, если оно выполнимо (т. е. выполняется хотя бы при одной оценке $\alpha$ предметных переменных в области интерпретации $M$).

\section{Тавтологии алгебры предикатов}
Любая тавтология алгебры высказываний является тавтологией алгебры предикатов. Более того, тавтологии алгебры высказываний дают возможность легко получать тавтологии алгебры предикатов с помощью следующего очевидного результата.

{\it Лемма 1.} Если $\Phi(X_1, \dots, X_n)$ -- тавтология алгебры высказываний, то для любых формул алгебры предикатов $\Phi_1, \dots, \Phi_n$ формула $\Phi(\Phi_1, \dots, \Phi_n)$ является тавтологией алгебры предикатов.

$\vDash \neg(X\land Y) \Leftrightarrow \neg X \lor \neg Y$ -- тавтология алгебры высказываний

$\vDash \neg(\Phi \lor \Psi) \Leftrightarrow \neg \Phi \lor \neg \Psi$ -- тавтология алгебры предикатов, если $\Phi, \Psi$ - формулы алгебры высказываний.

С другой стороны, в алгебре предикатов можно получить много принципиально новых тавтологий с помощью следующих свойств кванторов.

{\it Лемма 2.} Для любых формул $\Phi, \Psi$ следующие формулы явлюятся тавтологиями:
\begin{enumerate}
    \item $\neg(\forall x)\Phi \Leftrightarrow (\exists x)\neg\Phi$ \\ $\neg(\exists x)\Phi \Leftrightarrow (\forall x)\neg\Phi$ \\
    $(\forall x)\Phi \Leftrightarrow \neg(\exists x)\neg\Phi$ \\ $(\exists x)\Phi \Leftrightarrow \neg(\forall x)\neg\Phi$
    \item $(\forall x)(\forall y)\Phi \Leftrightarrow (\forall y)(\forall x)\Phi$ \\ $(\exists x)(\forall y)\Phi \Rightarrow (\forall y)(\exists x)\Phi$
    \item $(\forall x)(\Phi \land \Psi) \Leftrightarrow (\forall x)\Phi \land (\forall x)\Psi$ \\ $(\exists x)(\Phi \lor \Psi) \Leftrightarrow (\exists x)\Phi \lor (\exists x)\Psi$
    \item $(\forall x)(\Phi \pi \Psi) \Leftrightarrow (\forall x)\Phi \pi \Psi$, где $\pi$ -- символ одной из операций $\land, \lor$,
    \item $(\exists x)(\Phi \pi \Psi) \Leftrightarrow (\exists x)\Phi \pi \Psi$, где $\pi$ -- символ одной из операций $\land, \lor$, если в формулу $\Psi$ предметная переменная $x$ не входит свободно
\end{enumerate}

{\it Пример.} Рассмотрим $\Phi = P(x, y)$ и покажем \\
$\cancel{\vDash} (\forall y)(\exists z)P(x, y) \Rightarrow (\exists x)(\forall y)P(x, y)$. \\
Возьмём инт. $M = \mathbb{N}, P_M(x, y) = (y \leq x)$. Тогда \\
$\mathbb{N} \vDash (\forall y)(\exists x)P(x, y)$, т. к. для любого знач. $y = a$ найд. знач. $x$, для кот. $a \leq x$: \\
$\mathbb{N} \cancel{\vDash} (\exists x)(\forall y)P(x, y)$, т. к. это утверждает, что найдётся такое знач. $x = a$, что для всех знач $y = b$ вып. $b \leq a$. Это неверно.

С дизъюнкцией квантор общности переносить нельзя: $\cancel{\vDash} (\forall x)(\Phi(x)\lor\Psi(x)) \Leftrightarrow (\forall x)\Phi(x) \lor (\forall x)(\Psi(x))$

\section{Логическая равносильность двух формул}

\dftion Формулы алгебры предикатов  $\Phi, \Psi$ называются логически равносильными
, если результат примененния к ним логической операции эквивалентность $\Phi \Leftrightarrow \Psi$ является тавтологией.
В этом случае записывают $\Phi \equiv \Psi$ или $\Phi = \Psi$.
Таким образом, $\Phi = \Psi$ означает, что $\vDash \Phi \Leftrightarrow \Psi$.

\textbf{Теорема 1 (взаимосвязь между кванторами)}.
$\forall \Phi$:
\begin{equation*}
    (\forall x)(\forall y) \Phi = (\forall y)(\forall x)\Phi, (\exists x)(\exists y) \Phi = (\exists y)(\exists x)\Phi
\end{equation*}

С другой стороны, если в формулу $\Phi$ предметные переменные $x, y$ входят свободно, то равенство
$$(\forall y)(\exists x)\Phi = (\exists x)(\forall y)\Phi$$
не выполняется, так как в этом случае формула
$$(\forall y)(\exists x)\Phi \then (\exists x)(\forall y)\Phi$$
не является тавтологией.

\textbf{Теорема 2}. Пусть формула $\Phi(x)$ не содержит предметную переменную $y$ и формула $\Phi(y)$ получается из $\Phi(x)$ заменой всех свободных вхождений переменной $x$ на предметную переменную $y$.

Тогда формулы $(\forall x)\Phi(x)$ и $(\exists x)\Phi(x)$ будут логически равносильны соответственно формулам $(\forall y)\Phi(y)$ и $(\exists y)\Phi(y)$, то есть выполняются равенства:
$$(\forall x)\Phi(x) = (\forall y)\Phi(y)\text{ и }(\exists x)\Phi(x) = (\exists y)\Phi(y).$$

\textbf{Теорема 3 (законы де Моргана для кванторов)}. Для любой формулы $\Phi$ справедливы следующие утверждения:
\begin{itemize}
    \item $\lnot(\forall x)\Phi = (\exists x)\lnot \Phi$, $\lnot(\exists x)\Phi = (\forall x)\lnot \Phi$,
    \item $(\forall x)\Phi = \lnot(\exists x)\lnot \Phi$, $(\exists x)\Phi = \lnot(\forall x)\lnot \Phi$
\end{itemize}

\textbf{Теорема 4 (взаимосвязь кванторов с конъюнкцией и дизъюнкцией)}. Для любых формул $\Phi, \Psi$ справедливы следующие утверждения:
\begin{itemize}
    \item $(\forall x)(\Phi \land \Psi) = (\forall x)\phi \land (\forall x)\Psi$,
    \item $(\exists x)(\Phi \lor \Psi) = (\exists)\Phi \lor (\exists x)\Psi$
\end{itemize}

Если в формулу $\Psi$ переменная $x$ не входит свободно, то справедливы также утверждения:

$(\forall x)\Phi \pi \Psi = (\forall x)(\Phi \pi \Psi)$, $(\exists x)\Phi \pi \Psi = (\exists x)(\Phi \pi \Psi)$, где $\pi$ --- символ одной из операций $\land, \lor$.

\textbf{Теорема 6 (взаимосвязь кванторов с импликацией)}. Если в формулу $\Phi$ предметная переменая $x$ не входит свободно, то для любой формулы $\Psi$ справедливы следующие утверждения:

$(\forall x)(\Phi \then \Psi) = \Phi \then (\forall x)\Psi, (\exists x)(\Phi \then \Psi) = \Phi \then (\exists x)\Psi$.

Если же предметная переменная $x$ не входит свободно в формулу $\Psi$, то для любой формулы $\Phi$ справедливы утверждения:

$(\forall x)(\Phi \then \Psi) = (\exists x)\Phi \then \Psi$, $(\exists x)(\Phi \then \Psi) = (\forall x)\Phi \then \Psi$.

Спасибо Роберту за логическую равносильность формул


{\it Следствие 7.} Любая формула $\Phi$ представляетя в следующем виде:
\begin{equation*}
    \Phi = (K_1 x_1)\dots(K_n x_n)\Psi,
\end{equation*}
где $K_1,\dots,K_n$ --- некоторые кванторы и $\Psi$ --- формула без кванторов.

Таким образом, каждая формула $\Phi$ логически равносильна формуле $(K_1 x_1)\dots(K_n x_n)\Psi$, в которой все кванторы стоят в самом начале формулы и которая называется \textit{предварённой нормальной формой} (сокращённо ПНФ) формулы $\Phi$.


{\it Алгоритм} приведения формулы $\Phi$ к ПНФ:

1) преобразуем формулу $\Phi$ в эквивалентную ей формулу $\Phi'$, которая не содержит импликации и эквивалентности и в которой отрицание действует только на элементарные формулы

2) в $\Phi'$ все кванторы последовательно выносим вперёд по теореме 5, при этом кванторы общности выносятся из конъюнкции и квандоры существования выносятся из дизъюнкции, а для выноса кванторов общности из дизъюнкции и кванторов существования из конъюнкции переименовываем связанные переменные $x$ в новые переменные $y$, которые не входят в рассмотренную формулу.

{\underline{Пример.}} Найдём ПНФ для формулы $\Phi = (\exists x)(\forall y)P(x, y) \Rightarrow (\forall y)(\exists x)P(x, y) = \neg(\exists x)(\forall y)P(x, y) \lor (\forall y)(\exists x)P(x, y) = (\forall x)(\exists y)\neg P(x, y) \lor (\forall y)(\exists x)P(x, y)$ \\
Выполним замену $y \rightarrow u, x \rightarrow v$: \\
$= (\forall x)(\exists y)\neg P(x, y)\lor(\forall u)(\exists v)P(v, u) = (\forall x)(\exists y)(\neg P(x, y))\lor(\forall u)(\exists v)P(v, u) = (\forall x)(\exists y)(\forall u)(\exists v)\underset{\Psi}{\underbrace{(\neg P(x,y) \lor P(v, u))}}$. Мы получили ПНФ, так как формула $\Psi$ без кванторов

\section{Практика}

\underline{Задача 2.}  Выясните, справедливо ли следующее логическое следование: 
\begin{equation*}
    F \Rightarrow G, K \Rightarrow \neg H, H \lor \neg G \vDash F \Rightarrow \neg K
\end{equation*}
Решение. \\
$F \underset{\Phi_1} \Rightarrow G, K \underset{\Phi_2} \Rightarrow \neg H, H \lor \neg G \vDash F \underset \Phi \Rightarrow \neg K$ \\
Д-во от противного. \\
Предположим, чо это лог. след. не вып., для некот истин. зн. перем. $F, G, H, K$ вып.:
1) $F \Rightarrow G = 1$ \\
2) $K \Rightarrow \neg H = 1$ \\
3) $H \lor \neg G = 1$ \\
4) $F \Rightarrow K$ = 0 \\
Из случая 4) получаем $F = 1, \neg K = 0$, т. е. $K = 1$ \\
Из 1): $1 \Rightarrow G = 1$, $G = 1$ \\
Из 2): $1 \Rightarrow \neg H 1$, $\neg H = 1$, $H = 0$ \\
Из 3): $0 \lor \neg 1 = 0 \neq 1$ -- противоречит условию 3. Значит, наше предположение верно и логическое условие выполняется.


\underline{Задача 5.}

\subsection{Метод проверки тождественной истинности формул}

\underline{Метод 1. С помощью таблицы.} Тривиально.

\underline{Метод 2. Алгебраический метод.} Разбирался ранее.
%\underline{Метод 3. Алгоритм Квайна.} IMG_20240322_123816_397.jpg и IMG_20240322_123821_796.jpg


\underline{Задача.} С помощью алгоритма Квайна выясните, является ли тождественно истинной формула
\begin{equation*}
    \Phi = ((Y \Rightarrow Z) \land (X \Rightarrow V) \land (X \lor \neg Z)) \Rightarrow (\neg Y \lor V)
\end{equation*}
Строим дерево решений (см. конспект в тетради). Фиксируем $X = 1$:
\begin{equation*}
    (Y \Rightarrow Z) \land (1 \Rightarrow V) \land \cancel{(1 \lor \neg Z)} \Rightarrow (\neg Y \lor V)
\end{equation*}
\begin{equation}
    (Y \Rightarrow Z) \land V \Rightarrow (\neg Y \lor V)
\end{equation}
В (3.1) фиксируем $Y=1$:
\begin{equation*}
    (1 \Rightarrow Z) \land V \Rightarrow (0 \lor V)
\end{equation*}
\begin{equation}
    Z \land V \Rightarrow V
\end{equation}
В (3.2) фиксируем $V=1$:
\begin{equation*}
    Z \land 1 \Rightarrow 1
\end{equation*}
\begin{equation*}
    Z \Rightarrow 1 = 1
\end{equation*}
Верно. Теперь в (3.2) фиксируем $V=0$:
\begin{equation*}
    Z \land 0 \Rightarrow 0 = 1
\end{equation*}
Верно. \\ \\ Теперь в (3.1) фиксируем $Y=0$:
\begin{equation*}
    (0 \Rightarrow Z) \land V \Rightarrow (1 \lor V)
\end{equation*}
И так далее.