\section{Интерпретация формул алгебры предикатов}
\dftion \textbf{Область интерпретации} --- непустое множество $M$, которое является областью возможных значений всех предметных переменных.
\dftion \textbf{Оценка} предметных переменных --- $\alpha: X \to M$, где $X$ --- множество всех предметных переменных, $M$ --- область интерпретации.
\dftion \textbf{Выполнимость} формулы $\Phi$ в интерпретации $M$ при оценке $\alpha$ обозначается $M \vDash_\alpha \Phi$ и означает, что формула $\Phi$ истинна в интерпретации $M$ при оценке $\alpha$.

\dftion  $\Phi$ является:
\begin{itemize}
    \item \textit{общезначимой} (тождественно истинной, общезначимой), если $M \vDash_\alpha \Phi$ в любых интерпретациях $M$ при любых оценках $\alpha$. обозначается как $\vDash \Phi$;
    \item \textit{выполнимой}, если $M \vDash_\alpha \Phi$ в некоторой $M$ при некоторой $\alpha$
    \item \textit{опровержимой}, если в некоторой $M$ при некоторой $\alpha$ неверно, что $M \vDash_\alpha \Phi$
    \item \textit{тождественно ложной}, если в любой $M$ для любой $\alpha$ неверно, что $M \vDash_\alpha \Phi$
\end{itemize}

Любая тавтология алгебры высказываний ялвяется тавтологией алгебры предикатов.