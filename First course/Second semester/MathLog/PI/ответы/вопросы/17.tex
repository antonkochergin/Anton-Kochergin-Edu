\section{Унификация формул}
\dftion \textbf{Подстановки} --- отображения $\theta: X_S \to T_S$, где $X_S$ --- множество предметных переменных, $T_S$ --- множество термов. Обозначаются
$$\theta = \left(
\begin{matrix}
    x_1 & \dots & x_n \\
    t_1 & \dots & t_n
\end{matrix}
\right),$$
где $\forall x_i \in dom \, \theta \; \,  \theta(x_i) \neq x_i \then \theta \; t_i=\theta(x_i)$

Пусть $W=\{\Phi_1,\dots,\Phi_k\}$ --- множество атомарных формул, встречающихся в формулах из множества $S$. Подстановка $\theta$ называется \textit{унификатором множества формул} $W$, если $\theta(\Phi_1) = \dots = \theta(\Phi_k$).

Говорят, что множество атомарных формул $W$ \textit{унифицируемо}, если для него существует унификатор.