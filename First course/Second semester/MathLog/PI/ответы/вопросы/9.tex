\section{Кванторы и общности их существования, их действие на предикат. Свободны и связанные переменные}
\dftion $\forall$ --- квантор общности, $\exists$ --- квантор существования.

$n-1$-местный предикат $(\forall x_1)P(x_1,\dots,x_n)$ зависит от переменных $x_2,\dots,x_n$ и при значениях $x_2=a_2,\dots,x_n=a_n$ истинен тогда и только тогда, когда при любых значениях $x_1 = a_1 \in M$ высказывания $P(a_1,a_2,\dots,a_n)$ истинны.

$n-1$-местный предикат $(\exists x_1)P(x_1,\dots,x_n)$ зависит от переменных $x_2,\dots,x_n$ и при значениях $x_2=a_2,\dots,x_n=a_n$ истинен тогда и только тогда, когда существует такое значение $x_1 = a_1 \in M$, при котором высказывание $P(a_1,a_2,\dots,a_n)$ истинно.

\dftion \textbf{Связные} переменные --- переменные, которые находятся в области действия одного из кванторов. Все прочие переменные называются \textbf{свободными}
