\section{Полиномиальные сведения проблем}
Говорят, что машина Тьюринга $T$ имеет \textbf{полиномиальную временную сложность} $P(n)=n^k$ (<<время работы ограничено полиномом $P(N)$>>), если, обрабатывая вход $w$ длины $n$, $T$ делает не более $P(n)$ переходов и останавливается независимо от того, допущен вход или нет.

Основной метод доказательства того, что проблему $P_2$ нельзя решить за полиномиальное время (т. е. $P_2 \not\in \mathscr{P}$) состоит в сведении к ней за полиномиальное время такой проблемы $P_1$, что $P_1 \not\in P$. Такое преобразование языков называется \textit{полиномиальным сведением}.

\dftion Говорят, что язык $L$ является $\mathscr{NP}-\text{трудным}$, если для любого языка $L'$ из класса $\mathscr{NP}$ существует полиномиальное сведение языка $L'$ к языку $L$.

\dftion Говорят, что язык $L$ является $\mathscr{NP}-\textbf{полным}$, если он принадлежит классу $\mathscr{NP}$ и является $\mathscr{NP}\text{-трудным}$.

\textbf{Теорема 1}. Если проблема $P_1$ является $\mathscr{NP}$-трудной и существует полиномиальное сведение проблемы $P_1$ к проблеме $P_2$, то проблема $P_2$ также $\mathscr{NP}$-трудна.

\underline{Следствие}. Если проблема $P_1$ является $\mathscr{NP}$-полной и существует полиномиальное сведение проблемы $P_1$ к проблеме $P_2 \in \mathscr{NP}$, то проблема $P_2$ также $NP$-полна.