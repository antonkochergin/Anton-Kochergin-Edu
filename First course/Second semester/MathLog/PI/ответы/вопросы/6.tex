\section{Логическое следование формул. Методы доказательства логического следования формул}
\dftion \textbf{Логическое следование} обозначается символом $\vDash$ и означает, что при подстановке в некоторые формулы $\Phi_1, \dots, \Phi_n$ вместо их переменных конкретных значений $A_1, \dots, A_n$ из истинности $\Phi_1(A_1,\dots,A_n),\dots,\Phi_n(A_1,\dots,A_n)$ следует истинность высказывания $\Phi(A_1,\dots,A_n)$. $\Phi$ при этом называется \textbf{следствием}, а $\Phi_1, \dots, \Phi_n$ --- \textbf{посылками}.

Частными случаями следования является:
\begin{itemize}
    \item $\vDash \Phi$ или $1 \vDash \Phi$ --- тавтология;
    \item $\Phi \vDash$ или $\Phi \vDash 0$ --- противоречие
\end{itemize}

Основные правила логического следования:
\begin{itemize}
    \item Правило отделения: $\Phi, \Phi \then \Psi \vDash \Psi$;
    \item Правило контрапозиции: $\Phi \then \Psi \vDash \lnot \Psi \then \lnot \Phi$;
    \item Правило цепного следования: $\Phi \then \Psi, \Psi \then \Xi \vDash \Phi \then \Xi$;
    \item Правило перестановки посылок: $\Phi \then (\Psi \then \Xi) \vDash \Psi \then (\Phi \then \Xi)$
\end{itemize}

\begin{enumerate}
    \item \textbf{Прямой метод}
    \item \textbf{Алгебраический метод} предполагает вывод тождественно истинного высказывания 1 путём равносильных преобразований
    \item \textbf{Метод Квайна} предполагает построение дерева путём подстановки значений из множества $\{0,1\}$ вместо очередной переменной на очередной единице глубины дерева.
    \item \textbf{Метод редукции} предполагает принятие высказывания за ложное и доказательства противоречивости путём пошагового восстановления исходных значений переменных
    \item \textbf{Метод семантических таблиц}
    \item \textbf{Метод резолюций} предполагает построение резолютивного вывода нуля из КНФ
\end{enumerate}