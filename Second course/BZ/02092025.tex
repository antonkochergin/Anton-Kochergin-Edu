\documentclass{article}
\usepackage{amsmath}
\usepackage{array}
\usepackage{fancyvrb}
\usepackage{minted}
\usepackage{graphicx}
\usepackage[hidelinks]{hyperref}
\usepackage[utf8]{inputenc}
\usepackage[T2A]{fontenc}
\usepackage[russian]{babel}

\begin{document} 


 \textbf{02.09.2025 Лекция}
 
 ОБЖ - это интегральная учебная дисциплина, которая опирается на знания в
 области экологии, физики, химии, математики, ряда технических дисциплин (теория
 машин и механизмов, теория прочности и надежности, технология производств),
 физиологии и психологии человека, токсикологии и гигиены, социологии и
 экономики.

 Основные разделы:

 1) теоретические основы и методолгия безопасности деятельности человека

 2) производственная (технологическая безопасность)

 3) экологическая безопасность
 
 4) безопасность в чрезвычайных ситуациях


 В СГУ есть отдел экологической безопасности

 \begin{itemize}
    \item здоровье и уровень безопасности человека
    \item анализ процессов взаимодействия в системах "Человек, Машина, Окружающая среда"
    \item подходы к проектированию техники и технологических процессов, принятие решений с учетом современных требований экологии и безопасности
    \item прогнозирование в условиях ЧС

\end{itemize}

\textbf{Жизнедеятельность} - повседневная деятельность и отдых, способ существования человека

\textbf{Среда обитания} - окружающая человека среда, обусловленная совокупностью факторов (физических, химических, биологических, информационных, социальных), способных оказывать прямое или косвенное немедленное или отдаленное воздействие на жизнедеятельность человеку его здоровья и потомства

система "человек --- среда обитания"

\textbf{БЖД} - наука о комфортном и безопасном взаимодействии человека и окружающей среды

\textbf{Биосфера} - природная область распространения жизни на Земле, включающая
нижний слой атмосферы, гидросферу и верхний слой литосферы, не испытавших
техногенного взаимодействия

\textbf{Техносфера} - регион биосферы в прошлом, преобразованный людьми с
помощью прямого или косвенного воздействия технических средств в целях
наилучшего соответствия своим материальным и социально-экономическим
потребностям. 

\textbf{Бытовая среда} - сумма факторов, воздействующих на человека в быту (коммунальная
гигиена, гигиена питания, гигиена детей и подростков).

\textbf{Производственная среда} - совокупность факторов, воздействующих на человека в
процессе трудовой деятельности

\textbf{Безопасность в природной среде} - одна из отраслей экологии. Экология изучает
закономерности взаимодействия организмов с окружающей средой. Среда обитания
неразрывно связана с понятием "биосфера".

\textbf{Безопасность жизнедеятельности} - область знаний, в которой изучаются опасности, угрожающие человеку, закономерности с их проявлениями и способы защиты 

\textbf{Вредный фактор} - воздействие на человека, которое в определнных условиях приводит к постепенному ухудшению состояния здоровья заболеванию или снежению работоспособности.

\textbf{Опасный фактор} - воздействие на человека, которое в определенных условиях приводит к трамве или другому резкому ухудшению здоровья

Пространство, в котором постоянно действует или периодически возникают опасные и вредные факторы, принято называть опасной зоной

Опасные зоны по пространственным характеристикам могут быть локальными и развернутыми, а по времени - постоянными и временными.

\textbf{Источники опасности} - материальные объекты, носители опасных факторов.

\textbf{Безопасность} - состояние деятельности, при котором с определнной вероятностью исключено проявления опасности.

\textbf{Риск} - количесттвенная оценка опасности, определяется как частота или вероятность возникновения неблагоприятного с точки зрения безопасноти события.

Риск - отношение тех или иных нежелательных последствий в единицу времени к возможнму числу событий.

Концепция приемлего риска - риск, при котором защитные мероприятия позволяют поддерживать доступный уровень безопансости.



Безопансоть жизнедеятельности рассматривает

\begin{itemize}
    \item безопасность в бытовой среде
    \item безопансость в производственной сфере
    \item безопаснсоть жизнедеятельности в городской среде
    \item безопаснсоть в окружающей среде
    \item чрезвычайные ситуации мирного и военного времени
\end{itemize}

Задачи БЖД

\begin{itemize}
    \item идентификация опасности техносферы
    \item разработке и использованию средств защиты от опасностей
    \item их непрерыному контролю и мониторингу в техносфере
    \item обучению работающих и населения основам защиты от опасностей
    \item разработке мер по ликвидации последствий проявления опасностей
\end{itemize}

Техника безопасности - система организационных мероприятий и техниических средств, предотвращающих воздействие на работающих опасных факторов.

Охрана труда - система законодательных атковЮ социально-экономических, организационных, технических, гигиенических и лечебно-профилактических мероприятий и средств, обеспечивающих безопасность, сохранение здоровья и работоспособности человека в процессе труда.

Производственная санитария - система организационных мероприятий и технических средств, предотвращающих или уменьшающих воздейтсиве на работающих вредных производственных факторов.

\end{document}
